\hypertarget{implementazione}{%
\chapter{Implementazione di altre funzionalità}\label{capitolo-3---implementazione}}

\hypertarget{codice-per-controllare-se-un-utente-ha-trovato-un-match}{%
\section{Codice per controllare se un utente ha trovato un
match}\label{codice-per-controllare-se-un-utente-ha-trovato-un-match}}

L'utilità di tale controllo è quella di aggiornare l'interfaccia utente,
ed in particolare sostituire il \texttt{FindMatchFragment} con il
\texttt{MatchFoundFragment} una volta che un utente ha trovato un utente di ruolo opposto compatibile.

\begin{lstlisting}[caption=Controllo per vedere se l'utente è in un match o meno]
public void checkIfGiverOrTakerIsFound() {
    boolean isGiver = "giver".equals(userRole);
    app.me(true, me -> {
        Car car = me.getOwner().get(0);
        if (car.getMatches() != null) {
            List<Match> matches = car.getMatches().stream().sorted(Comparator.reverseOrder()).collect(Collectors.toList());
            Match match = matches.get(0);
            if ("running".equals(match.getStatus())) {
                //Match found!!
                try {
                    FragmentUtils.replaceFragment(isGiver, activity, R.id.findMatchFragmentContainer, MatchFoundFragment.class, false);
                } catch (Exception e) {
                    e.printStackTrace();
                }
            } else {
                checkIfGiverOrTakerIsFoundHandler.postDelayed(checkIfGiverOrTakerIsFoundRunnable, TIMEOUT);
            }
        }
    });
}
\end{lstlisting}

La funzione \texttt{checkIfGiverOrTakerIsFound()} effettua una chiamata
alle API per ottenere la lista dei match e li ordina per data di
creazione, con i match più recenti che occupano i primi posti della
lista.

Prende poi in considerazione il match in cima alla lista, ovvero il più
recente, e guarda se è un match in stato di \texttt{RUNNING}.

Se questo è vero, allora vuol dire che l'utente ha trovato un altro
utente di ruolo compatibile e quindi è all'interno di un match in corso.
Viene effettuata la sostituzione del corrente \texttt{FindMatchFragment}
con il \texttt{MatchFoundFragment} relativo al ruolo dell'utente.

\hypertarget{codice-per-controllo-stato-del-match}{%
\section{Codice per controllo stato del
match}\label{codice-per-controllo-stato-del-match}}

L'utilità di tale controllo è quella di verificare se l'utente è ancora
all'interno di un match in corso oppure se quest'ultimo è stato
terminato.

\begin{lstlisting}[caption=Controllo per verificare se un match è completato]
private void matchCompletedListener() {
    app.me(true, me -> {
        Car car = me.getOwner().get(0);
        if (car.getMatches() != null) {
            Match match = car.getMatches().stream().sorted(Comparator.reverseOrder()).collect(Collectors.toList()).get(0);
            switch (match.getStatus()) {
                case "success":
                    exitMatch(MatchStatusDialog.MatchStatus.SUCCESS);
                    break;
                case "unsuccess-taker":
                    exitMatch(MatchStatusDialog.MatchStatus.UNSUCCESS_TAKER);
                    break;
                case "unsuccess-giver":
                    exitMatch(MatchStatusDialog.MatchStatus.UNSUCCESS_GIVER);
                    break;
                case "waiting":
                    MatchStatusDialog.setMatchStatus(MatchStatusDialog.MatchStatus.WAITING);
                    matchCompletedHandler.postDelayed(matchCompletedRunnable, TIMEOUT);
                    break;
                case "running":
                    MatchStatusDialog.setMatchStatus(MatchStatusDialog.MatchStatus.RUNNING);
                    matchCompletedHandler.postDelayed(matchCompletedRunnable, TIMEOUT);
                    break;
                case "deleted":
                    exitMatch(MatchStatusDialog.MatchStatus.DELETED);
                    break;
                case "expired":
                    exitMatch(MatchStatusDialog.MatchStatus.EXPIRED);
                    break;
                default:
                    exitMatch(null);
                    Log.e(TAG, "Error while checking match state");
                    break;
            }
        }
    });
}
\end{lstlisting}


Il codice appena mostrato viene inizializzato sia per il taker che per
il giver nella callback \texttt{onCreateView} del \emph{fragment.}

\begin{lstlisting}[caption=Inizializzazione dell'handler e del runnable]
matchCompletedHandler = new Handler();
matchCompletedRunnable = this::matchCompletedListener;
matchCompletedHandler.post(matchCompletedRunnable);
\end{lstlisting}


La funzione \texttt{matchCompletedListener()} effettua una chiamata alle
API per ottenere la lista dei match e li ordina per data di creazione,
con i match più recenti che occupano i primi posti della lista.

Controlla lo stato del match in cima e, se tale match ha uno stato
diverso da \texttt{RUNNING} o da \texttt{WAITING}, e quindi non è più in
corso, rimuove il \texttt{MatchFoundFragment} portando l'utente nella
pagina principale, e resetta la mappa in modo da eliminare eventuali
elementi obsoleti tramite la funzione \texttt{exitMatch()}.

\begin{lstlisting}[caption=Funzione per ripristinare la UI al termine del match]
private void exitMatch(MatchStatusDialog.MatchStatus matchStatus) {
    if (takerPositionUpdateHandler != null) {
        takerPositionUpdateHandler.removeCallbacks(takerPositionUpdateRunnable);
    }
    MatchStatusDialog.setMatchStatus(matchStatus);
    matchCompletedHandler.removeCallbacks(matchCompletedRunnable);
    takerToGiverRoute.forEach(Polyline::remove);
    resetMapBounds();
    removeAllCarPins();
    FragmentManager fragmentManager = activity.getSupportFragmentManager();
    fragmentManager.popBackStack();
}
\end{lstlisting}

In caso invece il match ha uno stato di \texttt{RUNNING} o di
\texttt{WAITING}, vuol dire che è ancora in corso, e quindi la funzione
viene richiamata nuovamente dopo un tempo definito in nella
variabile \texttt{TIMEOUT}, che in questo caso vale \texttt{2000ms}, in
modo da effettuare nuovamente il controllo.

\hypertarget{codice-per-tempo-stimato-di-incontro-tra-taker-e-giver}{%
\subsection{Codice per tempo stimato di incontro tra taker e
giver}\label{codice-per-tempo-stimato-di-incontro-tra-taker-e-giver}}

\hypertarget{percorso-tra-taker-e-giver}{%
\subsection{Percorso tra taker e
giver}\label{percorso-tra-taker-e-giver}}

\hypertarget{conversione-da-coordinate-a-indirizzo-in-background}{%
\subsection{Conversione da coordinate a indirizzo in
background}\label{conversione-da-coordinate-a-indirizzo-in-background}}

Per la conversione della posizione da coordinate, e quindi oggetto
formato da una coppia di valori latitudine e longitudine, ad una stringa
che rappresenta l'indirizzo, viene utilizzata la funzione
\texttt{getFromLocation} della classe \texttt{Geocoder}.

Tale metodo, se invocato in maniera semplice, viene eseguito in maniera
sincrona nello stesso thread in cui viene eseguita l'interfaccia utente.

Essendo un'operazione che può richiedere del tempo, invocarlo sullo
stesso thread all'interno di un'interfaccia grafica potrebbe portare ad
una serie di bug, di rallentamenti e di errori del tipo ANR
(\emph{Application Non Responding).}

Per eseguire il metodo in maniera asincrona, questo viene eseguito in un
thread in background mediante un \emph{Service}.

Fonte:
\url{https://android-doc.github.io/training/location/display-address.html}

Viene definita una classe \texttt{FetchAddressIntentService} che estende
la classe \texttt{IntentService}.

Per essere riconoscibile dal dispositivo, il \emph{service} deve essrere
definito nel file \texttt{AndroidMainfest.xml.}

\begin{lstlisting}
<service
    android:name=".services.FetchAddressIntentService"
    android:exported="false"/>
\end{lstlisting}


Nella classe \texttt{FetchAddressIntentService} viene ridefinito il
metodo \texttt{onHandleIntent} come segue:
\begin{lstlisting}
@Override
protected void onHandleIntent(@Nullable Intent intent) {
    List<Address> addresses = null;
    Location location = intent.getParcelableExtra(GCConstants.LOCATION_DATA_EXTRA);
    Geocoder geocoder = new Geocoder(getApplicationContext(), Locale.getDefault());
    receiver = intent.getParcelableExtra(GCConstants.RECEIVER);
    String errorMessage = "";
    try {
        addresses = geocoder.getFromLocation(
                location.getLatitude(),
                location.getLongitude(),
                // Get just a single address.
                1);
    } catch (IOException ioException) {
        // Catch network or other I/O problems.
        errorMessage = "Service not available";
        onHandleIntent(intent);
    } catch (IllegalArgumentException illegalArgumentException) {
        // Catch invalid latitude or longitude values.
        errorMessage = "Invalid LatLng used";
    }
    // Handle case where no address was found.
    if (addresses == null || addresses.isEmpty()) {
        if (errorMessage.isEmpty()) {
            errorMessage = "No address found";
        }
        deliverResultToReceiver(GCConstants.FAILURE_RESULT, errorMessage);
    } else {
        String address = addresses.get(0).getAddressLine(0);
				//Take only the address name and number, without the city and postal code.
        String result = address.split(",")[0] + ", " + address.split(",")[1]; 
        deliverResultToReceiver(GCConstants.SUCCESS_RESULT, result);
    }
}
\end{lstlisting}

Tale metodo viene eseguito quando una classe A invia dei dati tramite un
\texttt{Intent} esplicito alla classe
\texttt{FetchAddressIntentService}.

All'interno del campo extra dell'Intent vengono inseriti due oggetti:

\begin{itemize}

\item
  Un oggetto \texttt{Location}
\item
  Un oggetto \texttt{Receiver}
\end{itemize}

La costruzione appena citata avviene nel metodo
\texttt{startAddressIntentService()}.

\begin{lstlisting}
private void startAddressIntentService() {
    Intent fetchAddressIntent = new Intent(getActivity(), FetchAddressIntentService.class);
    fetchAddressIntent.putExtra(GCConstants.RECEIVER, new AddressResultReceiver(null));
    Location location = getPointerLocation();
    fetchAddressIntent.putExtra(GCConstants.LOCATION_DATA_EXTRA, location);
    requireActivity().startService(fetchAddressIntent);
}
\end{lstlisting}


Nel metodo \texttt{FetchAddressIntentService} vengono quindi estratti i
dati riguardanti la posizione da convertire in indirizzo, questa viene
convertita tramite il metodo \texttt{Geocoder.getFromLocation} e la
stringa rappresentante l'indirizzo viene inviata al \texttt{Receiver}
tramite il metodo \texttt{deliverResultToReceiver}.

\begin{lstlisting}
private void deliverResultToReceiver(int resultCode, String message) {
    Bundle bundle = new Bundle();
    bundle.putString(GCConstants.RESULT_DATA_KEY, message);
    receiver.send(resultCode, bundle);
}
\end{lstlisting}


Il \texttt{Receiver} in questo caso viene implementato come classe
interna alla classe \texttt{ChangeParkPositionFragment.}

\begin{lstlisting}
class AddressResultReceiver extends ResultReceiver {

    public AddressResultReceiver(Handler handler) {
        super(handler);
    }

    @Override
    protected void onReceiveResult(int resultCode, Bundle resultData) {
        if (MainScreenActivity.isEditMode()) { //Change UI only if the user is still inside this UI
            requireActivity().runOnUiThread(() -> {
                String address = resultData.getString(GCConstants.RESULT_DATA_KEY);
                confirmNewPark.setClickable(true);
                confirmNewPark.setBackgroundColor(getResources().getColor(R.color.primaryGreen));
                newAddress.setText(address);
            });
        }
    }
}
\end{lstlisting}

Tale \texttt{Receiver} effettua un \emph{override} del metodo
\texttt{onReceiveResult} il quale viene invocato quando il Receiver
riceve un risultato da un \emph{service} e definisce le operazioni che
devono essere effettuate con tale risultato.

In questo caso la classe \texttt{AddressResultReceiver}, che estende
dalla classe \texttt{ResultReceiver} che a sua volta estende da
\texttt{Receiver}, riceve il risultato inviato da
\texttt{FetchAddressIntentService} e lo utilizza per mostrare su schermo
l'indirizzo del nuovo parcheggio quando l'utente si trova all'interno
della \emph{edit mode,} e sta quindi spostando il marker del parcheggio
nella potenziale nuova posizione.

La posizione puntata dal marker viene quindi convertita per essere di
volta in volta visualizzata dall'utente sotto forma di indirizzo.

\hypertarget{controllo-di-match-pre-esistenti-per-il-ripristino-dellinterfaccia}{%
\section{Controllo di match pre-esistenti per il ripristino
dell'interfaccia}\label{controllo-di-match-pre-esistenti-per-il-ripristino-dellinterfaccia}}

Quando l'applicazione viene avviata, viene effettuato un controllo per
verificare se l'utente sia in protagonista all'interno di un match con
status \texttt{RUNNING} o \texttt{WAITING}, in maniera da ripristinare
la corretta schermata.

\hypertarget{utilituxe0-di-tale-controllo}{%
\subsubsection{Utilità di tale
controllo}\label{utilituxe0-di-tale-controllo}}

Si ipotizzi che tale controllo non venga effettuato, e si ipotizzi che
l'utente prema sul pulsante ``Cerca posto'' per cercare un \emph{giver}
con cui scambiare un posto. L'applicazione allora mostrerà la schermata
di ricerca di un posto (Figura 1).

Si ipotizzi però che, per qualche motivo, l'applicazione venga chiusa, e
quindi l'interfaccia proseguirà a chiamare le opportune \emph{callbacks}
per rimuovere i \emph{fragments.}

Al successivo riavvio dell'applicazione, essa, non effettuando nessun
controllo sullo stato di un eventuale match, mostrerà la schermata
iniziale, anche se l'utente è attualmente in ricerca di un posto. Questo
disallinamento dell'interfaccia del sistema con lo stato del sistema può
portare ad una serie di gravi bug.

\hypertarget{codice-per-effettuare-il-controllo}{%
\subsubsection{Codice per effettuare il
controllo}\label{codice-per-effettuare-il-controllo}}

\begin{lstlisting}
private void resumePendingMatchUI() {
		//Effettuo una chiamata alle API per ottenere le informazioni circa l'utente
    app.me(true, me -> {
        if (me.getOwner().size() == 0) {
            return;
        }
        Car car = me.getOwner().get(0);
        if (car.getMatches() != null) {
            List<Match> matches = car.getMatches().stream().sorted(Comparator.reverseOrder()).collect(Collectors.toList());
            for (Match match : matches) {
                //Controllo se l'utente abbia dei match in sospeso
                if ("waiting".equals(match.getStatus())) {
                    //Se l'utente è taker nel match in sospeso
                    if (car.getCid().toString().equals(match.getTakerCid())) {
                        try {
                            //Rimpiazzo il frammento con il FindMatchFragment da taker
                            FragmentUtils.replaceFragment(false, getActivity(), R.id.carFragment, FindMatchFragment.class, true, true);
                        } catch (Exception e) {
                            e.printStackTrace();
                        }
                        //Se l'utente è giver nel match in sospeso
                    } else if (car.getCid().toString().equals(match.getGiverCid())) {
                        try {
                            //Rimpiazzo il frammento con il FindMatchFragment da giver
                            FragmentUtils.replaceFragment(true, getActivity(), R.id.carFragment, FindMatchFragment.class, true, true);
                        } catch (Exception e) {
                            e.printStackTrace();
                        }
                    }
                    break;
                    //Controllo se l'utente abbia dei match in corso
                } else if ("running".equals(match.getStatus())) {
                     //Se l'utente è taker nel match in corso
                    if (car.getCid().toString().equals(match.getTakerCid())) {
                        try {
                            //Rimpiazzo il frammento con il  MatchFoundFragment da taker
                            FragmentUtils.replaceFragment(false, getActivity(), R.id.carFragment, MatchFoundFragment.class, true, true);
                        } catch (Exception e) {
                            e.printStackTrace();
                        }
                        //Se l'utente è giver nel match in corso
                    } else if (car.getCid().toString().equals(match.getGiverCid())) {
                        try {
                            //Rimpiazzo il frammento con il  MatchFoundFragment da giver
                            FragmentUtils.replaceFragment(true, getActivity(), R.id.carFragment, MatchFoundFragment.class, true, true);
                        } catch (Exception e) {
                            e.printStackTrace();
                        }
                    }
                  break;
                }
            }
        }
    });
}
\end{lstlisting}

Tale controllo avviene nella \emph{callback} \texttt{onCreateView()} del
\texttt{CarFragment}.

La funzione \texttt{resumePendingMatchUI()} effettua una chiamata alle
API per ottenere la lista dei match relativi all'utente e la ordina in
modo analogo alla funzione \texttt{matchCompletedListener()} prima
citata.

Controlla poi il match in cima alla lista per vedere se è un match con
stato \texttt{RUNNING} o \texttt{WAITING} e, se una di queste condizioni
è verificata, ripristina rispettivamente il \texttt{MatchFoundFragment}
o il \texttt{FindMatchFragment}.

In caso nessuna delle due condizioni è verificata la funzione non
effettua nulla.