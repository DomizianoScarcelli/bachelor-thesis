\chapter{Conclusioni}

\section{Sommario}

Nel \autoref{introduzione} sono stati analizzati gli obiettivi che l'applicazione su cui si è svolto il lavoro vuole risolvere, ed in che modo promette di farlo.

Nel \autoref{tecnologie-utilizzate} sono state mostrate le tecnologie utilizzate all'interno del tirocinio, insieme ai concetti fondamentali per lo sviluppo di applicazioni Android.

Nel \autoref{progettazione} è stata discussa la fase di progettazione della UI (User Interface) e della UX (User Experience). La prima è volta a comprendere come organizzare gli elementi mostrati sullo schermo in maniera da rendere l'applicazione facile da utilizzare; la seconda è volta a capire in che modo strutturare l'intero sistema per rendere l'esperienza che l'utente ha con quest'ultimo il più piacevole possibile.

Sono stati quindi mostrati i prototipi delle interfacce che sono state sviluppate, descrivendo le relative funzionalità, e sono stati documentati i test di usabilità effettuati, elencando i problemi individuati e le soluzioni che sono state proposte, fino all'arrivo di un'interfaccia completa.

Nel \autoref{implementazione} sono stati approfonditi i dettagli implementativi circa le interfacce e funzionalità progettate nel \autoref{progettazione}, mostrando le parti di codice significative che hanno portato alla realizzazione delle funzionalità desiderate.

Il tirocinante, alla fine del lavoro, ha maturato le seguenti capacità:
\begin{itemize}
    \item Progettazione di un prototipo di interfaccia;
    \item Implementazione di un'interfaccia all'interno di Android a partire da un prototipo;
    \item Sviluppo di funzionalità in ambito Android in linguaggio Java;
    \item Effettuare chiamate API e gestire le risposte.
\end{itemize}

\pagebreak
\section{Sviluppi Futuri}
Dall'inizio del tirocinio alla fine di quest'ultimo all'interno dell'applicazione sono state inserite numerose funzionalità, tuttavia l'applicazione è ancora nelle prime fasi di sviluppo e quindi sono molteplici quelle che ancora possono essere implementate.

Le funzionalità principali a mancare per quanto riguarda la gestione dei match sono:
\begin{itemize}
    \item La possibilità di un utente di cercare un posto in una zona differente da quella in cui si trova attualmente;
    \item La possibilità di un utente di cercare un posto in un orario differente da quello attuale, in modo da prenotarsi il posto per un momento futuro della giornata;
    \item La possibilità di un utente di essere Giver e Taker contemporaneamente.
\end{itemize}

A parte le prime due funzionalità che sono abbastanza autodescrittive, in questa sezione si descriveranno i dettagli e le complicazioni riguardanti la possibile implementazione della terza funzionalità.

Si ipotizzi che le prime due funzionalità elencate sopra già esistano all'interno dell'applicazione, e si ipotizzi la seguente situazione:
\begin{enumerate}
    \item Sia A un utente attualmente parcheggiato;
    \item A effettua un "Cerca posto" in un'altra zona;
\end{enumerate}

L'utente diventa quindi un Taker e raggiunge il Giver B associato per occupare il suo posto. Essendo l'utente A però parcheggiato, ed essendo egli un Taker, non può effettuare un "Lascia posto" prima di raggiungere l'utente B. 

Questo vuol dire che il posto da lui lasciato non verrà immediatamente occupato da un eventuale Giver C.

È possibile risolvere questo problema dissociando le azioni di "Cerca posto" e "Lascia posto", in modo che queste siano eseguibili indipendentemente l'una dall'altra.

Un utente A parcheggiato che intende effettuare un "Cerca posto", potrebbe quindi effettuare prima un "Lascia posto" per trovare un Taker B compatibile che occupi il suo attuale posto.
Effettuerà il "Cerca posto" nell'attesa che il Taker arrivi nella sua posizione, in modo da trovarsi già un utente Giver C compatibile. Una volta che il Taker B sia arrivato nella posizione dell'utente A e questi si siano scambiati i posti, l'utente A raggiungerà l'utente C per scambiare il posto auto con lui.

